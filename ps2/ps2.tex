%%This is a very basic article template.
%%There is just one section and two subsections.
\documentclass{article}

\usepackage{amsmath}
\usepackage{amsfonts}
\usepackage{parskip}
\usepackage{cleveref}
\usepackage{xcolor} 

\setlength{\parskip}{0cm} 

\title{TTIC 31230 Problem Set 2 \\ Win 2017}

\author{Hao Jiang}
\begin{document}

\maketitle

\section*{Problem 1}
Using the concentration inequality for gradient estimation mentioned in the
slides,
\begin{align*}
||\nabla_w\ell_{\text{train}}(w) - \nabla_w \ell_{\text{generalize}}(w)|| \leq
\frac{b(1+\sqrt(2ln(1/\delta)))}{\sqrt{N}}
\end{align*}
with probability $1-\delta$
\section*{Problem 2}
\subsection*{a}
The experiment result is shown in \Cref{tab:2a}. The following things can be
observed.
\begin{itemize}
  \item When batch size is small (10), large learning rate (0.37) tends
  to increase training loss
  \item When batch size is large (100), larger learnining rate ($\eta^*$)
  reduces training loss
  \item For the same step size, larger batch size has a smaller training loss
\end{itemize}

\begin{table}
\centering
\begin{tabular}{c|c|c}
\texttt{Setting} & \texttt{Training Loss} & \texttt{Test Accuracy} \\
\hline
B = 10, $\eta = \eta^*(B)$ & 0.0094 & 0.9965 \\
\hline
B = 10, $\eta = 0.37$ & 0.1013 & 0.9787\\
\hline
B = 100, $\eta = \eta^*(B)$ & 0.0107& 0.9977\\
\hline
B = 100, $\eta = 0.37$ & 0.0223 & 0.9940\\
\end{tabular}
\caption{Experiment Result for Problem 2a}
\label{tab:2a}
\end{table}

\textbf{Conclusion}: Optimal learning rate varies with batch size and there
seems no simple rule that determines a best learning rate given a batch size.
\subsection*{b}

\begin{table}
\centering
\begin{tabular}{c|c|c}
\texttt{Setting} & \texttt{Training Loss} & \texttt{Test Accuracy} \\
\hline
B = 50, $\eta = 0.37$ & 0.0101 & 0.9969 \\
\hline
B = 50, $\eta = \eta^*$ & 0.0098& 0.9968\\
\hline
B = 10, $\eta = 0.37$ & 0.0756 &  0.9799\\
\hline
B = 10, $\eta = \eta^*$ & 0.0076& 0.9970\\
\hline
B = 100, $\eta = 0.37$ & 0.0200 & 0.9950 \\
\hline
B = 100, $\eta = \eta^*$ & 0.0108& 0.9965\\
\end{tabular}
\caption{Experiment Result for Problem 2b}
\label{tab:2b}
\end{table}

\subsection*{c}

\subsection*{d}
Most important thing is training speed. Smaller batch size is much much slower
\end{document}
