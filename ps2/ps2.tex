%%This is a very basic article template.
%%There is just one section and two subsections.
\documentclass{article}

\usepackage{amsmath}
\usepackage{amsfonts}
\usepackage{parskip}
\usepackage{cleveref}
\usepackage{xcolor} 

\setlength{\parskip}{0.2cm}

\title{TTIC 31230 Problem Set 2 \\ Win 2017}

\author{Hao Jiang}
\begin{document}

\maketitle

\section*{Problem 1}
\begin{align*}
&||\nabla_w\ell(w,x,y)|| < b
\implies \forall i,n, |(\nabla_w \ell(w,x_i,y_i))_n -
(\ell_{\text{generalize}}(w))_n| < 2b
\end{align*}
Using the concentration inequality for gradient estimation mentioned in the
slides,
\begin{align*}
||\nabla_w\ell_{\text{train}}(w) - \nabla_w \ell_{\text{generalize}}(w)|| \leq
\frac{2b\left(1+\sqrt{2\ln(1/\delta)}\right)}{\sqrt{N}}
\end{align*}
with probability $1-\delta$
\section*{Problem 2}
\subsection*{a}
The experiment result of 2a is shown in \Cref{tab:2a}. The following things can be
observed.
\begin{itemize}
  \item The provided optimal learning rate $\eta^*$ yields a better test 
  accuracy on batch size 10 and 100, but not on batch size 50.
  \item For the same learning rate(0.37), larger batch size does not always yield
  better performance result
\end{itemize}

\begin{table}
\centering
\begin{tabular}{c|c|c}
\texttt{Setting} & \texttt{Training Loss} & \texttt{Test Accuracy} \\
\hline
B = 10, $\eta = \eta^*(B)$ & 0.0099 & 0.9969 \\
\hline
B = 10, $\eta = 0.37$ & 0.0817 & 0.9780\\
\hline
B = 50, $\eta = \eta^*(B)$ & 0.0097 & 0.9976 \\
\hline
B = 50, $\eta = 0.37$ & 0.0086 & 0.9980 \\
\hline
B = 100, $\eta = \eta^*(B)$ & 0.0110& 0.9977\\
\hline
B = 100, $\eta = 0.37$ & 0.0213 & 0.9935\\
\end{tabular}
\caption{Experiment Result for Problem 2a}
\label{tab:2a}
\end{table}

\textbf{Conclusion}: There is no single best learning rate for different 
batch sizes. The learning rate should be setup as a function relying on batch
size to provide best performance.

\subsection*{b}
The experiment result of 2b is demonstrated in \Cref{tab:2b}. In the case of
momentum, $\eta^*$ does always yield a better result. When batch size is small,
the difference is more obvious. When batch size is larger, the difference become
less.

Similar to what is observed in standard SGD, for the given constant learning rate
$\eta= 0.37$, batch size 50 gives the best performance.
\begin{table}
\centering
\begin{tabular}{c|c|c}
\texttt{Setting} & \texttt{Training Loss} & \texttt{Test Accuracy} \\
\hline
B = 10, $\eta = 0.37$ & 0.1193 & 0.9745\\
\hline
B = 10, $\eta = \eta^*$ & 0.0076& 0.9970\\
\hline
B = 50, $\eta = 0.37$ & 0.0098 & 0.9970 \\
\hline
B = 50, $\eta = \eta^*$ & 0.0097& 0.9975\\
\hline
B = 100, $\eta = 0.37$ & 0.0199 & 0.9947 \\
\hline
B = 100, $\eta = \eta^*$ & 0.0108& 0.9965\\
\end{tabular}
\caption{Experiment Result for Problem 2b}
\label{tab:2b}
\end{table}

\subsection*{c}
The result of Adam algorithm is demonstrated in \Cref{tab:2c}.

For batch size 50, minimal train loss and test accuracy can be 
obtained around $\eta = 0.0016$.

For batch size 100, the minimal training loss appears at $\eta = 0.0015$,
but the highest test accuracy appears at $\eta = 0.0014$

For batch size 10, the minimal training loss appears at $\eta = 0.0014$, 
and the highest test accuracy appears at $\eta = 0.00145$

In addition, it can also be noticed that a smaller batch size tends
to yield a higher training loss and lower test accuracy.
\begin{table} 
\centering
\begin{tabular}{c|c|c}
\texttt{Setting} & \texttt{Training Loss} & \texttt{Test Accuracy} \\
\hline
B = 50, $\eta = 0.0014$ & 0.0110 & 0.9974\\
\hline
B = 50, $\eta = 0.0015$ & 0.0099 & 0.9973\\
\hline
B = 50, $\eta = 0.0016$ & 0.0088 & 0.9979\\
\hline
B = 50, $\eta = 0.0017$ & 0.0090 & 0.9973\\
\hline
B = 100, $\eta = 0.0013$ & 0.0116 & 0.9972\\
\hline
B = 100, $\eta = 0.00135$ & 0.0104 & 0.9976\\
\hline
B = 100, $\eta = 0.0014$ & 0.0093 & 0.9979\\
\hline
B = 100, $\eta = 0.00145$ & 0.0096 & 0.9978\\
\hline
B = 100, $\eta = 0.0015$ & 0.0089 & 0.9978\\
\hline
B = 100, $\eta = 0.0016$ & 0.0100 & 0.9975\\ 
\hline
B = 10, $\eta = 0.0013$ & 0.0223& 0.9925\\
\hline
B = 10, $\eta = 0.00135$ & 0.0136& 0.9946\\
\hline
B = 10, $\eta = 0.00138$ & 0.0190& 0.9938\\
\hline
B = 10, $\eta = 0.0014$ & 0.0124& 0.9946\\
\hline
B = 10, $\eta = 0.00145$ & 0.0143 & 0.9955 \\
\hline
B = 10, $\eta = 0.0015$ & 0.0191& 0.9937\\
\hline
B = 10, $\eta = 0.0016$ & 0.0229 & 0.9918\\ 
\end{tabular}
\caption{Experiment Result for Problem 2c}
\label{tab:2c}
\end{table}

\subsection*{d}
One of the most obvious observation is the relationship between batch size and
running time. The running time will in general decrease when the batch size 
increase. However, the decrease rate will become slower when batch size is too
large. When computation is applied to a larger batch group, the efficiency of 
numpy's vector operations will save some time, but the average result need to be
computed on a larger group. Thus there should exist an optimal batch size for 
specific given problem.

Also, as mentioned in previous section, a smaller batch size tends to lead to 
both a larger training loss and a smaller test accuracy. This is possibly because
in a smaller batch group, outliers can have large impact to the average gradient
and lead to inaccurate result.

Finally, larger training loss in general yield to a worse test accuracy. But a
smaller training loss does not always means a higher test accuracy. This is due to
the skew in the sampling process when choosing training set and test set from the
distribution that includes all data.
\end{document}
