%%This is a very basic article template.
%%There is just one section and two subsections.
\documentclass{article}

\usepackage{amsmath}
\usepackage{amsfonts}
\usepackage{parskip}
\usepackage{xcolor}

\setlength{\parskip}{0cm}

\title{TTIC 31230 Problem Set 1 \\ Win 2017}

\author{Hao Jiang}
\begin{document}

\maketitle

\section*{Problem 1}
\subsection*{a}
Using chain rule, we have
\begin{align*}
\frac{\partial \ell}{\partial x} = \frac{\partial \ell}{\partial
g}\frac{\partial g}{\partial x} +\frac{\partial \ell}{\partial h}\frac{\partial
h}{\partial x}
\end{align*}

Using backpropagation, we first compute the result for the hidden layer (the
layer of $y$ and $z$), having the value of $\delta_y = \frac{\partial
\ell}{\partial y}$ and $\delta_z = \frac{\partial \ell}{\partial z}$.
and for the input layer,
\begin{align*}
\frac{\partial \ell}{\partial x} &= g'(y)\delta_y + h'(z)\delta_z\\
&= \frac{\partial \ell}{\partial
g}\frac{\partial g}{\partial x} +\frac{\partial \ell}{\partial h}\frac{\partial
h}{\partial x}
\end{align*}
This result is equal to the result from chain rule and shows that
backpropagation correctly compute the derivative.
\subsection*{b}

For second derivative
\begin{align*}
\frac{\partial^2 \ell}{\partial x^2} &= 
\frac{\partial}{\partial x}(\frac{\partial \ell}{\partial g}\frac{\partial g}{\partial x} +\frac{\partial \ell}{\partial h}\frac{\partial
h}{\partial x}) \\
&= \frac{\partial^2 \ell}{\partial g^2}\left(\frac{\partial g}{\partial
x}\right)^2 +\frac{\partial \ell}{\partial g}\frac{\partial^2 g}{\partial x^2}
+\frac{\partial^2 \ell}{\partial h^2}\left(\frac{\partial h}{\partial
x}\right)^2 +\frac{\partial \ell}{\partial h}\frac{\partial^2 h}{\partial x^2}
\end{align*}

This is not equivalent to the backpropagation result
\begin{align*}
\frac{\partial^2 \ell}{\partial g^2}\frac{\partial^2 g}{\partial x^2} +
\frac{\partial^2 \ell}{\partial h^2}\frac{\partial^2 h}{\partial x^2}
\end{align*}
\section*{Problem 2}
Softmax function
\begin{align*}
Y = [y_1,y_2,\ldots,y_n] &= S([x_1,x_2,\ldots,x_n]) = S(X)\\
&= \frac{1}{\sum_{i=1}^n
\exp(x_i)}[\exp(x_1),\exp(x_2),\ldots,\exp(x_n)]
\end{align*}
Let $Z = \sum_{i=1}^n\exp(x_i)$, the derivative
\begin{align*}
\frac{\partial y_j}{\partial x_i} =
\left\{
\begin{array}{cc}
\frac{\exp(x_j)}{Z}-\frac{(\exp^2(x_j)}{Z^2} = y_i(1-y_i) & i = j \\
-\frac{\exp(x_i)\exp(x_j)}{Z^2} = -y_iy_j & i \neq j \\
\end{array}
\right.
\end{align*}

When using backpropagation to compute $\nabla x_i$, we know $\nabla x_i =
\sum_{i=k}^n \nabla y_k \frac{\partial y_k}{\partial x_i}$, thus
\begin{align*}
\nabla X = J\nabla Y
\end{align*}
where $J_{ij} = \frac{\partial y_j}{\partial x_i}$.

\begin{align*}
J\nabla Y &= \left(\left[ 
\begin{array}{cccc}
y_1 & & & \\
 &y_2 & &\\
& & \ddots & \\
& & & y_n\\
\end{array}
\right] - YY^T\right)\nabla Y \\
&= \left[\begin{array}{c}y_1\nabla y_1\\y_2\nabla
y_2\\\vdots\\y_n\nabla y_n\end{array}\right] - YY^T\nabla Y\\
&= \left[\begin{array}{c}y_1\nabla y_1\\y_2\nabla
y_2\\\vdots\\y_n\nabla y_n\end{array}\right] - \texttt{gvdot}*Y\\
&=\left[\begin{array}{c}y_1(\nabla y_1 - \texttt{gvdot})\\y_2(\nabla
y_2-\texttt{gvdot})\\\vdots\\y_n(\nabla
y_n-\texttt{gvdot})\end{array}\right]\\
\end{align*}

This is an element-wise multiplication of $Y$ and $(\nabla Y - \texttt{gvdot})$.

In the EDF implementation, \texttt{self.value *
(self.grad - gvdot)} is wrong. The correct code should be
\texttt{np.multiply(self.value, self.grad - gvdot)}

\section*{Problem 3}
\end{document}
